% !TEX program = xelatex
\documentclass[]{resume-openfont}

\pagestyle{fancy}
\resetHeaderAndFooter

\setlength{\parindent}{1.5em}
\setlength{\parskip}{1.5em}
\renewcommand{\baselinestretch}{1.3}

%--------------------------------------------------------------
% Convenience command - make it easy to fill template

% Create job position command. Parameters: company, position, location, when
\newcommand{\resumeHeading}[4]{\runsubsection{#1}\descript{\\
#2}\hfill\location{#3 | #4}\fakeNewLine}

% Create education heading. Parameters: Name, degree, location, when
\newcommand{\educationHeading}[4]{\runsubsection{#1}\hspace*{\fill} \location{#3
 #4}\\
\textbf{#2}\fakeNewLine}

\newcommand{\teachingHeading}[4]{\runsubsection{#1}\hspace*{\fill} \location{#3
 #4} \textbf{#2}\fakeNewLine}

% Create project heading. Parameters: Name, link, Tech stack
\newcommand{\projectHeading}[3]{\Project{#1}{#2}
\descript{#3}\\}

\newcommand{\projectHeadingWithDate}[4]{\Project{#1}{#2}
\descript{#3 #4}\\}

% Parameters: courses
\newcommand{\courseWork}[1]{\textbf{Curso:} #1}

% Parameters: courses
\newcommand{\teacherAssistant}[1]{\textbf{Ayudante Alumno:} #1}
 
%--------------------------------------------------------------
\begin{document}

%--------------------------------------------------------------
%     Perfil
%--------------------------------------------------------------
\newcommand{\yourName}{OCAMPO, Nicolás Benjamín}
\newcommand{\yourWebsite}{Blog}
\newcommand{\yourWebsiteLink}{https://benjaminocampo.netlify.app/}
\newcommand{\yourEmail}{nicolasbenjaminocampo@gmail.com}
\newcommand{\yourPhone}{~+54~(387)~15469~0467}
\newcommand{\githubUserName}{benjaminocampo}
\newcommand{\linkedInUserName}{benjamin-ocampo}

\begin{center}
    \Huge \scshape \latoRegular{\yourName}
\end{center}

%--------------------------------------------------------------
%     Datos Personales
%--------------------------------------------------------------

\noindent
\begin{minipage}{0.6\textwidth}

    \section{Datos Personales}

    \sectionsep

    \large{\textbf{Apellido y Nombre:} \yourName
    \vspace*{0.6em}

    \textbf{Lugar de Nacimiento:} Salta Capital, Argentina
    \vspace*{0.6em}

    \textbf{Fecha de Nacimiento:} 15 de Julio de 1999
    \vspace*{0.6em}

    \textbf{Estado Civil:} Soltero
    \vspace*{0.6em}

    \textbf{Teléfono:} \yourPhone
    \vspace*{0.6em}

    \textbf{Correo electrónico:}
    \href{mailto:\yourEmail}{\underline{\yourEmail}}
    \vspace*{0.6em}

    \textbf{LinkedIn:}
    \href{https://www.linkedin.com/in/\linkedInUserName}{\underline{linkedIn/\linkedInUserName}}
    \vspace*{0.6em}

    \textbf{Sitio Web:} \href{\yourWebsiteLink}{\underline{\yourWebsiteLink}}
    \vspace*{0.6em}

    \textbf{GitHub:}
    \href{https://github.com/\githubUserName}{\underline{github/\githubUserName}}}
\end{minipage}
\hfill
\begin{minipage}{0.3\textwidth}\begin{flushright}
\end{flushright}
\end{minipage}%

%--------------------------------------------------------------
%     Educación
%--------------------------------------------------------------
\section{Educación}

\educationHeading{Diplomatura en Ciencia de Datos, Aprendizaje Automático y sus
Aplicaciones}{Facultad de Matemática, Astronomía, Física, y Computación,
Universidad Nacional de Córdoba}{Córdoba, Argentina |}{Mar 2021 - Actualidad}

\educationHeading{Licenciatura en Ciencias de la Computación}{Facultad de
Matemática, Astronomía, Física, y Computación, Universidad Nacional de Córdoba
\\
\faMortarBoard Promedio general: 9.58. \\
\faMortarBoard Promedio histórico de la carrera en los últimos 5 años:
8.10.}{Córdoba, Argentina |}{Ene 2017, Actualidad}

Fecha esperada para defensa del título de grado: Diciembre 2021. \\
Tema de la tesis: Uso de Word Embeddings para Grounding Heuristico en Planning Clásico.

\educationHeading{Analista en Computación}{Facultad de Matemática, Astronomía,
Física, y Computación, Universidad Nacional de Córdoba \\
    \faMortarBoard Promedio General: 9.56}{Córdoba, Argentina |}{Ene 2017
- Dic 2019}

\educationHeading{Escuela Secundaria}{Instituto de Educación Media Dr. Arturo
Oñativia \\
\faMortarBoard Promedio General: 9.22}{Salta, Argentina}{Ene 2011 - Dic 2016}


%--------------------------------------------------------------
%     Courses Possibly Relevant to the Position
%--------------------------------------------------------------

\vspace*{2em}

\section{Cursos Realizados}

\runsubsection{Materias de la carrera de grado donde obtuve una nota de 10 en la
Facultad de Matemática, Astronomía, Física, y Computación (FaMAF), Universidad
Nacional de Córdoba}

    \textbf{Matemática Discreta II.} Algoritmos de grafos. Flujo Maximal. Código
    Cíclicos. Problemas P y NP.

    \textbf{Computación Paralela.} Técnicas de optimización de programas.
    
    \textbf{Redes Neuronales.} Sistemas dinámicos. Modelado de neuronas reales y
    artificiales.

    \textbf{Probabilidad y Estadística.} Análisis y Visualización de datos.
    Variables aleatorias Test de hipótesis.

    \textbf{Modelos y Simulación.} Números aleatorios. Generación de variables
    aleatorias. Simulación de procesos estocásticos.

    \textbf{Análisis Matemático I} Funciones continuas. Valores extremos de
    funciones en intervalos cerrados. Derivadas. Extremos relativos. Reglas de
    L'Hopital. Derivadas sucesivas. Gráfico de funciones. Antiderivadas.

    \textbf{Análisis Matemático II.} Métodos de Integración, sucesiones y
    subsucesiones. Series numéricas. Series de potencia. Series de Taylor y
    polinomios. Geometría Analítica: rectas y planos en $R^n$. Funciones de
    varias variables, derivadas parciales, derivadas direccionales, y gradientes.
    Integrales múltiples.

    \textbf{Análisis Numérico.} Análisis de error absoluto y relativo. Redondeo
    y truncamiento. Sistemas de punto flotante. Solución de ecuaciones no
    lineales: bisección, newton, secante y métodos de punto fijo. Interpolación
    polinomial y splines. Cuadrados mínimos. Integración numérica. Sistemas de
    ecuaciones lineales.

    \textbf{Algoritmos y Estructuras de Datos II.} Representación de datos en
    memoria. Estructura de datos. Tipos abstractos de datos y sus
    implementaciones. Punteros. Algoritmos voraces. Divide y venceras. Recursión
    y backtracking. Programación Dinámica. Complejidad Algoritmica.

\runsubsection{Cursos optativos que participé en la Escuela de Verano RIO 2020,
Facultad de Ciencias Exactas, Físico-Químicas, y Naturales (FCEFyN), Universidad
Nacional de Río Cuarto, Argentina}

    \textbf{Enfoques Clásicos y Neuronales a la Minería de Texto.}
    Análisis teórico-práctico de información no estructurada obtenida a partir
    de textos provenientes de la interacción con el Procesamiento del Lenguaje
    Natural, Recuperación de la Información, y Machine Learning.

    \textbf{Software Testing.}
    Técnicas y conceptos de verificación de programas tales como Fuzzing,
    Symbolic Execution, Bounded Model Checking, entre otras.

\runsubsection{Cursos de docencia que participé en la Diplomatura Aprender a
Enseñar, Universidad Nacional de Córdoba}

    \textbf{Diplomatura Aprender a enseñar.} Capacitación sobre los roles de un
    ayudante alumno, tutor, y docente universitario y como llevarlos acabo de
    manera efectiva.

\section{Participación en Seminarios}

\educationHeading{Segmentación de Imágenes Médicas Usando Redes
Neuronales}{Organizado por el Grupo de Grandes Redes Sociales y Semánticas.
Duración: 1,30hs.}{}{FaMAF}

\educationHeading{Aprendizaje  Automático Sobre Datos Encriptados}{Organizado
por el Grupo de Grandes Redes Sociales y Semánticas. Duración: 1,30hs.}{}{FaMAF}

\educationHeading{Integrando Interpretabilidad y Eficiencia en la Detección
Anticipada de Riesgos en las Redes Sociales}{Organizado por el Grupo de Grandes
Redes Sociales y Semánticas. Duración: 1,30hs.}{}{FaMAF}

\educationHeading{Aplicaciones de Eye-tracking en Salud Visual y
Mental}{Organizado por el Grupo de Grandes Redes Sociales y Semánticas.
Duración: 1,30hs.}{}{FaMAF}

\educationHeading{El Futuro del Desarrollo Mobile}{Organizado por la escuela de
verano RIO 2020. Duración: 2hs.}{}{FCEFyN}

\educationHeading{Machine Learning — Soluciones y Aplicaciones en Environment
Business}{Organizado por la empresa Futit Services en la escuela de verano RIO
2020. Duración: 2hs.}{}{FCEFyN}

\educationHeading{Business Intelligence — Realidad y Experiencia}{Organizado por
la empresa Trimix en la escuela de verano RIO 2020. Duración: 2hs.}{}{FCEFyN}

\educationHeading{La Evolución de las APIs: GraphQL}{Organizado por la empresa
WeDevelop en la escuela de verano RIO 2020. Duración: 2hs.}{}{FCEFyN}

%-------------------------------------------------------------
%     Experiencia Docente
%--------------------------------------------------------------

\section{Experiencia Docente}
\teachingHeading{Ayudante Alumno}{}{FaMAF |}{Mar 2020 - Actualidad}

\textbf{Redes y Sistemas Distribuidos.} Preparación de material extra que
facilite la comprensión del material de estudio. Planificación de reuniones
virtuales con grupos de estudiantes para responder preguntas y dudas de la
materia.

\textbf{Sistemas Operativos.} Preparación de material extra que
facilite la comprensión del material de estudio. Planificación de reuniones
virtuales con grupos de estudiantes para responder preguntas y dudas de la
materia.

\textbf{Algoritmos y Estructuras de Datos I.} Solución a obstaculos didácticos y
cognitivos que afecten el progreso de los estudiantes. Sugerencias y comentarios
en la corrección de proyectos de laboratorios.

\textbf{Paradigmas de la Programación.} Planificación de reuniones
virtuales con grupos de estudiantes para responder preguntas y dudas de la
materia.

\textbf{Curso de Nivelación.} Preparación de clases online bajo un cronograma
estipulado. Respuestas de dudas por aula virtual.

\teachingHeading{Tutor}{}{FaMAF |}{Sep 2020 - Jun 2021}

Preparación de tutoriales online para introducir nuevos estudiantes a la vida
universitaria. Organización de eventos para complementar el plan de estudio de
los estudiantes por medio de conversaciones, juegos, y actividades.

\teachingHeading{Ayudante Alumno}{}{Instituto Técnico Superior Córdoba,
Argentina |}{Sep 2020 - May 2021}

\textbf{Verificación de Programas.} Proveer asistencia a estudiantes para
preparar exposiciones orales.

%--------------------------------------------------------------
%     Proyectos de Inteligencia Artificial y Programación
%--------------------------------------------------------------

\section{Pasantías}
\teachingHeading{Desarrollo de Software en Machine Learning}{}{SimTLiX |}{Oct 2021 - Actualidad}

Mantenimiento e implementación de nuevas funcionalidad en modelos neuronales
para la segmentación del húmero y escapula en imagenes médicas.

%--------------------------------------------------------------
%     Proyectos de Inteligencia Artificial y Programación
%--------------------------------------------------------------
\section{Proyectos de Programación (Inteligencia Artificial y Otros)}

\projectHeading{Categorización de Publicaciones de Mercado Libre
Libre}{https://github.com/benjaminocampo/Mentoria}{} \textbf{Desarrollado en la
Diplomatura en Ciencia de Datos, Aprendizaje Automático y sus Aplicaciones,
FaMAF.} \\
    \textbf{Lenguajes de programación y herramientas utilizadas: Python, Keras,
    Pandas, Scikit-learn, y NLTK.} \\
    \textbf{Duración: 9 meses (En desarrollo).}

    Implementación de un clasificador que predice la categoría de un producto dado
    el título de su publicación por medio de datos otorgados por la empresa
    Mercado Libre para el MeLi Challenge 2019.

\projectHeading{Optimización de Ecuaciones Navier
    Stokes}{https://gitfront.io/r/user-1529562/1df7f0690a8d6eeffcb1cffeb1c3a408084afd08/navierstokes-mirror/}{}
    \textbf{Desarrollado en la materia Computación Paralela de la Licenciatura
    en Ciencias de la Computación, FaMAF.} \\
    \textbf{Lenguajes de programación y herramientas utilizadas: C, C++, Python,
    CUDA, Intrinsics, y ISPC.} \\
    \textbf{Duración: 6 meses.}

    Aplicación de un simulador de dinámica de fluidos usando métodos numéricos.
    Implementación por medio de técnicas de bajo nivel que maximizen la performance.
    Desarrollo de scripts en Python para obtener, registrar y producir métricas.

\projectHeading{Blog Personal}{https://benjaminocampo.netlify.app/}{}
    \textbf{Actividad Extracurricular} \\
    \textbf{Lenguajes de programación y herramientas utilizadas: ReactJS,
    GatsbyJS, y Javascript.} \\
    \textbf{Duración: 3 meses.}

    Blog personal donde comparto mis experiencias, ideas, y tutoriales.

\vspace{3em}

\projectHeading{Estimación de Precios de Viviendas en Melbourne}{https://github.com/benjaminocampo/DataCuration}{}
    \textbf{Desarrollado en la materia Curación y Visualización de datos de la
    Diplomatura en Ciencia de Datos, Aprendizaje Automático y sus Aplicaciones,
    FaMAF.} \\
    \textbf{Lenguajes de programación y herramientas utilizadas: Python, y
    Pandas.} \\
    \textbf{Duración: 2 meses.}

    Exploración y curación de datos de un dataset dado por una competencia de
    Kaggle de viviendas que fueron puestas en venta en Melbourne, Australia
    durante el 2016 y 2017. Preparación de una matriz de datos con el fin de
    utilizar modelos de aprendizaje supervisados y estimar el precio de
    viviendas no antes registradas.

\projectHeading{Uso de Redes Neuronales como Modelos Predictivos y
    Generativos}{https://github.com/benjaminocampo/NeuralNetworks}{}
    \textbf{Desarrollado en la materia Redes Neuronales de la Licenciatura en
    Ciencias de la Computación, FaMAF.} \\
    \textbf{Lenguajes de programación y herramientas utilizadas: Python, y
    Pytorch.} \\
    \textbf{Duración: 2 meses.}

    Análisis de modelos Lotka Volterra que simulan un entorno predador-presa y
    como dos especies interactuan. Implementación de ecuaciones Integrate and
    Fire para el modelado de pulsos electricos neuronales transmisores de
    información. Uso de autoencoders variacionales para preprocesamiento y
    generación de imagenes.

\projectHeading{Simulación de un Sistema de Cajeros ATM
System}{https://github.com/benjaminocampo/atm_simulator}{} \textbf{Desarrollado
en la materia Modelos y Simulación de la Licenciatura en Ciencias de la
Computación, FaMAF.} \\
    \textbf{Lenguajes de programación y herramientas utilizadas: Python.} \\
    \textbf{Duración: 1 més.}

    Implementación de un simulador estocástico de eventos discretos de llegada de
    clientes a un sistema de cola en cajeros automáticos.

%--------------------------------------------------------------
%     Experiencia en Investigación
%--------------------------------------------------------------
\section{Experiencia en Investigación}

\teachingHeading{Use de Word Embeddings para Grounding Heurístico en Planning
Clásico}{}{FaMAF |}{Jul 2021 - Actualidad}

    \textbf{Directores: Dr. M. Dominguez y Dr. C. Areces, FaMAF}

    Uso de técnicas de machine learning que permitan guiar el proceso de
    grounding en problemas de planning clásico.

%--------------------------------------------------------------
%     Actividades de Extensión
%--------------------------------------------------------------
\section{Servicio a la Comunidad}

\teachingHeading{Acompañamiento Tecnológico a Sectores Vulnerables}{}{FaMAF}{Apr
2021 - Actualidad}

    Reacondicionamiento e instalación de software necesario en computadoras,
    laptops, y tablets, con el fin de ser donadas a niños, jovenes, y adultos
    con dificultades en el acceso de recursos tecnológicos.
%--------------------------------------------------------------
%     Actividades Extracurriculares
%--------------------------------------------------------------
\section{Actividades Extracurriculares}

\teachingHeading{Escuela de Verano RIO 2020}{}{FCEFyN |}{Feb 2020}

Escuela de Verano ubicada en la Universidad Nacional de Río Cuarto donde
participé en cursos de Ciencias de la Computación, seminarios, y actividades
extracurriculares.

\teachingHeading{Torneo Argentino de Programación}{}{ACM International Collegiate
Programming Contest |}{Sep 2019}

Competencia en grupo de tres integrantes donde representé a la FaMAF en la
resolución de un conjunto de problemas algorítmicos en el menor tiempo posible.

\teachingHeading{Training Camp Argentina}{}{FaMAF |}{Jul 2019}

Curso intensivo de dos semanas para desarrollar habilidades de resolución de
problemas en competencias de programación.

\teachingHeading{Campamento Científico Expedición Ciencia}{}{Neuquén, Argentina |}{Feb
2016}

Campamento vocacional con el fin de obtener un primer acercamiento a la
exploración y pensamiento científico.

\educationHeading{Carrera de Composición Musical: Trayecto Artístico
Profesional}{Escuela Superior de Música de la Provincia de Salta Nº 6003 José Lo
Giúdice}{Salta, Argentina |}{Mar 2012 - Dic 2014}
%--------------------------------------------------------------
%     Other Skills
%--------------------------------------------------------------
\section{Otras Habilidades}

\paragraph{Idiomas:}
{\begin{minipage}{10cm}
    \begin{tabular}{|l||c|c|c|} \hline
                 &  {Lectura}    &  {Habla}   &  {Escritura} \\ \hline \hline
    Español (Lengua madre) &  Muy bien &  Muy bien & Muy bien \\
    Inglés      & Muy bien   & Muy bien   & Muy bien  \\ \hline
    \end{tabular}
    \end{minipage}} \\
\paragraph{Programación:}{Python, Scala, C, C++, Javacript, Java, Haskell.} \\
\paragraph{Tecnologías:}{Git, Pandas, NLTK, Sklearn, Pytorch, Keras, SQL,
MongoDB, \LaTeX, CUDA, Intrinsics, OpenMP, ISPC, ReactJS, GatsbyJS.} \\
\paragraph{Pasatiempos:}{Yoga, musculación, clases de salsa y bachata, tocar la guitarra.}

\sectionsep
\end{document}